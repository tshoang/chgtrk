% \iffalse meta-comment
% 
% chgtrk.ins
% 
% Copyright (C) 2017--2018 by Thai Son Hoang
% <T dot S dot Hoang at ecs dot soton dot ac dot uk>
% --------------------------------------------------------------------
% 
% This file may be distributed and/or modified under the
% conditions of the LaTeX Project Public License, either version 1.3c
% of this license or (at your option) any later version.
% The latest version of this license is in:
% 
%      http://www.latex-project.org/lppl.txt
% 
% and version 1.3c or later is part of all distributions of LaTeX 
% version 2008/05/04 or later.
% 
% This work has the LPPL maintenance status "author-maintained".
% 
% The Current Maintainer of this work is T.S. Hoang.
%
% This work consists of the files chgtrk.dtx, chgtrk.ins,
% the derived file chgtrk.sty, the generated documentation
% chtrk.pdf, and some sample documents.
% 
% \fi
% 
% \iffalse
%<lstEventB>\NeedsTeXFormat{LaTeX2e}\relax
%<lstEventB>\ProvidesPackage{chgtrk}
%<lstEventB>    [2018/06/26 v0.1 Package for changes tracking] 
% 
%<*driver> 
\documentclass[a4paper]{ltxdoc}
\usepackage{chgtrk}
\usepackage{hyperref}
\EnableCrossrefs
% ^^A\CodelineIndex
\PageIndex
\RecordChanges

\begin{document}
\DocInput{chgtrk.dtx}
\end{document}
%</driver>
% \fi
% 
% \CheckSum{0}
% 
% \CharacterTable
% {Upper-case    \A\B\C\D\E\F\G\H\I\J\K\L\M\N\O\P\Q\R\S\T\U\V\W\X\Y\Z
% Lower-case    \a\b\c\d\e\f\g\h\i\j\k\l\m\n\o\p\q\r\s\t\u\v\w\x\y\z
% Digits        \0\1\2\3\4\5\6\7\8\9
% Exclamation   \!     Double quote  \"     Hash (number) \#
% Dollar        \$     Percent       \%     Ampersand     \&
% Acute accent  \'     Left paren    \(     Right paren   \)
% Asterisk      \*     Plus          \+     Comma         \,
% Minus         \-     Point         \.     Solidus       \/
% Colon         \:     Semicolon     \;     Less than     \<
% Equals        \=     Greater than  \>     Question mark \?
% Commercial at \@     Left bracket  \[     Backslash     \\
% Right bracket \]     Circumflex    \^     Underscore    \_
% Grave accent  \`     Left brace    \{     Vertical bar  \|
% Right brace   \}     Tilde         \~}
% 
% 
% \changes{v0.1}{2018/06/26}{Initial version}
% 
% \GetFileInfo{chgtrk.sty}
% 
% \DoNotIndex{\\}
% \DoNotIndex{\DeclareOption}
% \DoNotIndex{\ProcessOptions}
% \DoNotIndex{\RequirePackage}
% \DoNotIndex{\arabic}
% \DoNotIndex{\begin}
% \DoNotIndex{\csname,\csuse}
% \DoNotIndex{\def,\do,\dolistloop}
% \DoNotIndex{\end,\endcsname,\expandafter}
% \DoNotIndex{\hline}
% \DoNotIndex{\ifstrequal,\iftoggle,\item}
% \DoNotIndex{\label,\labelformat,\listadd}
% \DoNotIndex{\medskip}
% \DoNotIndex{\newcommand,\newcounter,\newenvironment,\newtoggle,\nomenclature}
% \DoNotIndex{\quad}
% \DoNotIndex{\renewcommand,\renewenvironment,\ref,\refstepcounter}
% \DoNotIndex{\setcounter,\small}
% \DoNotIndex{\textsf,\textwidth,\togglefalse,\toggletrue}
% \DoNotIndex{\value}
% \DoNotIndex{\xspace}
%
% \title{The \textsf{chgtrk} package\thanks{This document
% corresponds to \textsf{chgtrk}~\fileversion, dated~\filedate.}}
% \author{Thai Son Hoang \\ ECS, University of Southampton \\ \texttt{<T dot S dot Hoang at ecs dot
% soton dot ac dot uk>}}
% \date{June 26, 2018}
% 
% \maketitle
% 
% ^^A %%%%% Abstract %%%%%
% \begin{abstract}
%   This package was developed in order to ease the typesetting of
%   changes in \LaTeX{}.  It was developed at the University of Southampton.
% \end{abstract}
% 
% ^^A %%%%% Table of contents %%%%%
% \tableofcontents
% 
% ^^A %%%%% Introduction %%%%%
% \section{Introduction}
% 
% This package was developed in order to ease the typesetting of
% changes in \LaTeX{}.
% 
% ^^A %%%%% Usage %%%%%%
% \section{Usage}
% 
% Just like any other package, you need to request this package with a
% |\usepackage| command in the preamble. So in the simpler case, one
% just types
% 
% \indent |\usepackage{chgtrk}|
%
% \noindent to load the package.
%
% \subsection{Contributor Declaration}
% \label{sec:contr-decl}
% After package loading, contributors can be declared using the
% meta-command |\newCTcontributor{<Contributor>}|, which subsequently
% creates several ``personalised'' commands for the contributor.  For
% example, the following command
% 
% \indent |\newCTcontributor{Alice}| \newCTcontributor{Alice}
%
% \noindent creates the commands |\AliceAdd|, |\AliceChange|,
% |\AliceDelete|, |\AliceComment|, etc., for adding, changing,
% deleting texts, and comment accordingly. (More contributor's
% commands are described in Section~\ref{sec:contributor-commands}.)
% The name |Alice| will be used to identify the contributor of various
% annotations.
% 
% If a full name (e.g., with white space) is desirable for
% contributor's identification, the following form of
% |\newCTcontributor| should be used.
%
% \indent |\newCTcontributor{<Contributor>}{<Full Name>}|
%
% \noindent where |Full Name| is the full name used for contributor's
% identification and |Contributor| is the prefix of the contributor's
% commands.  As an example, the following command
% 
% \indent |\newCTcontributor[Bob]{Bob Smith}| 
% \newCTcontributor[Bob]{Bob Smith}
% 
% \noindent creates the commands |\BobAdd|, |\BobChange|,
% |\BobDelete|, |\BobComment|, etc., while using ``|Bob Smith|'' as
% the contributor's identification.
%
% \subsection{Contributor Commands}
% \label{sec:contributor-commands}
%
% \paragraph{Adding Texts.}  A contributor can annotate the newly
% added text with her own |\<Contributor>Add| command.  For example,
% the following shows the text that |Alice| adds using |\AliceAdd|
% command, i.e.,
%
% \indent |\AliceAdd{This is the text that Alice adds.}|
%
% \noindent produces
%
% \indent \AliceAdd{This is the text that Alice adds.}
%
% \noindent Similarly, the following
%
% \indent |\BobAdd{And this is the text that Bob adds.}|
%
% \noindent produces
%
% \indent \BobAdd{And this is the text that Bob adds.}
%
% \noindent Notice how ``|Bob Smith|'' is used as the contributor's
% identification.
%
% \paragraph{Changing Texts.}  A contributor can annotate the changed
% text with her own |\<Contributor>Change| command.  For example,
% the following shows the text that |Alice| changes using |\AliceChange|
% command, i.e.,
%
% \indent |\AliceChange{This is the old text}{This is the text that Alice changes.}|
%
% \noindent produces
%
% \indent \AliceChange{This is the old text}{This is the text that
% Alice changes.}
%
% \noindent Similarly, the following
%
% \indent |\BobChange{This is the old text}{And this is the text that Bob changes.}|
%
% \noindent produces
%
% \indent \BobChange{This is the old text}{And this is the text that Bob changes.}
%
% \noindent Notice how ``|Bob Smith|'' is used as the contributor's
% identification.
%
% \paragraph{Deleting Texts.}  A contributor can annotate the deleted
% text with her own |\<Contributor>Delete| command.  For example, the
% following shows the text that |Alice| deletes using |\AliceDelete|
% command, i.e.,
%
% \indent |\AliceDelete{This is the text that Alice deletes.}|
%
% \noindent produces
%
% \indent \AliceDelete{This is the text that Alice deletes.}
%
% \noindent Similarly, the following
%
% \indent |\BobDelete{And this is the text that Bob deletes.}|
%
% \noindent produces
%
% \indent \BobDelete{And this is the text that Bob deletes.}
%
% \noindent Notice how ``|Bob Smith|'' is used as the contributor's
% identification.
%
% \paragraph{Comment.}  A contributor can comment with her own
% |\<Contributor>Comment| command.  For example, the following shows
% |Alice|'s comments using |\AliceComment| command, i.e.,
%
% \indent |\AliceComment{This is Alice's comment.}|
%
% \noindent produces \AliceComment{This is Alice's comment.} the
% comment on the margin with a link to the place with the command is
% issued (here after the word ``produces'').  Similarly, the following
%
% \indent |\BobComment{And this is Bob's comment.}|
%
% \noindent produces \BobComment{And this is Bob's comment.} the
% comment on the margin. Notice how ``|Bob Smith|'' is used as the
% contributor's identification.
%
% \paragraph{In-line Comment.}  A contributor can comment in-line with
% her own |\<Contributor>InlineComment| command.  For example, the following
% shows |Alice|'s in-line comments using |\AliceInlineComment| command, i.e.,
%
% \indent |\AliceInlineComment{This is Alice's in-line comment.}|
%
% \noindent produces \AliceInlineComment{This is Alice's in-line
% comment.}
%
% \noindent Similarly, the following
%
% \indent |\BobInlineComment{And this is Bob's in-line comment.}|
%
% \noindent produces \BobInlineComment{And this is Bob's in-line
% comment.}
%
% \noindent Notice how ``|Bob Smith|'' is used as the contributor's
% identification.
%
% \paragraph{Comment Addressing Other Contributors.}  A contributor
% can comment addressing other contributors with her own
% |\<Contributor>CommentTo| command.  For example, the following shows
% |Alice|'s comments to |Bob| using |\AliceCommentTo| command, i.e.,
%
% \indent |\AliceCommentTo{Bob}{This is Alice's comment to Bob.}|
%
% \noindent produces \AliceCommentTo{Bob}{This is Alice's comment to
% Bob.} the comment on the margin with a link to the place with the
% command is issued (here after the word ``produces'').  Similarly,
% the following
%
% \indent |\BobComment{And this is Bob's comment.}|
%
% \noindent produces \BobComment{And this is Bob's comment.} the
% comment on the margin. Notice how ``|Bob Smith|'' is used as the
% contributor's identification.
%
% \StopEventually{
% \PrintChanges
% \PrintIndex
% }
% 
% ^^A %%%%% Implementation %%%%%
% \section{Implementation}
%
% ^^A %%% Package loading %%% 
% Our implementation is based on the |todonotes| () and |soul| packages.
% 
% \iffalse ^^A BEGIN Produce comments only in the resulting style file
%<chgtrk>
%<chgtrk>%%%%% BEGIN Package loading %%%%%
% \fi ^^A END Produce comments only in the resulting style file
%
%    \begin{macrocode}
\RequirePackage{soul}
\RequirePackage{todonotes}
\RequirePackage{etoolbox}
\RequirePackage{xargs}
%    \end{macrocode}
%
% \iffalse ^^A BEGIN Produce comments only in the resulting style file
%<chgtrk>%%%%% END Package loading %%%%%
%<chgtrk>
% \fi ^^A END Produce comments only in the resulting style file
%
% \begin{macro}{\newabbrev}
%   \changes{v0.0}{2013/04/22}{Macro created}
%   The |newabbrev| makes use of the worker macro |newfullabbrev| for
%   creating abbreviations macros.
% 
% \iffalse^^A BEGIN Produce comments only in the resulting style file
%<*chgtrk>

% Meta-macro to create abbreviation macros.
%
% Arguments:
% 1. (Optional) String to be used as macro
% 2. The abbreviation (also used as the macro if the optional argument
%    is empty)
% 3. The full string expansion.
%
% Usage:
% - \newabbrev{SME}{Small and Medium-sized Enterprise} will create
% macros \SME will expand as Small and Medium-sized Enterprise (SME)
% and \SMEs will expand as Small and Medium-sized Enterprises (SMEs).
% - \newabbrev[randd]{R\&D}{Research and Development} will create
% macros \randd will expand as Research and Development (R\&D)
% and \randds will expand as Research and Developments (R\&Ds).
%</chgtrk>
% \fi^^A END Produce comments only in the resulting style file
%    \begin{macrocode}
%   A wrapper for \ifstrequal to make sure that the first argument is properly expanded
\newcommand{\chgtrk@ifstrequal}{\expandafter\ifstrequal\expandafter}

% Create a new contributor, i.e., the set of macros for the
% contributor to add, delete, change, comment, and comment inline.
% Arguments:
% 1. The macro prefix string, (OPTIONAL) if empty then the contributor name will be used.
% 2. The contributor name
% Usage:
% - \CT@contributor{Susan} will create a new set of macros
% \SusanAdd, \SusanDelete, \SusanChange, \SusanComment,
% \SusanInlineComment, which will be expanded to \CT@add[Susan],
% \CT@delete[Susan], \CT@change[Susan], \CT@comment[Susan],
% \CT@inlinecommmented[Susan]
% - \CT@contributor[sls]{Susan Louise Smith} will create a new set of macros
% \slsAdd, \slsDelete, \slsChange, \slsComment,
% \slsInlineComment, which will be expanded to \CT@add[Susan Louise Smith],
% \CT@delete[Susan Louise Smith], \CT@change[Susan Louise Smith], \CT@comment[Susan Louise Smith],
% \CT@inlinecommmented[Susan Louise Smith]
\newcommand{\CT@contributor}[2][]{%
  \chgtrk@ifstrequal{#1}{}{%
    \expandafter\def\csname #2Add\endcsname{\CT@add[#2]}%
    \expandafter\def\csname #2Delete\endcsname{\CT@delete[#2]}%
    \expandafter\def\csname #2Change\endcsname{\CT@change[#2]}%
    \expandafter\def\csname #2Comment\endcsname{\CT@comment[#2]}%
    \expandafter\def\csname #2CommentTo\endcsname{\CT@commentto[#2]}%
    \expandafter\def\csname #2InlineComment\endcsname{\CT@inlinecomment[#2]}%
    \expandafter\def\csname #2InlineCommentTo\endcsname{\CT@inlinecommentto[#2]}%
  }{%
    \expandafter\def\csname #1Add\endcsname{\CT@add[#2]}%
    \expandafter\def\csname #1Delete\endcsname{\CT@delete[#2]}%
    \expandafter\def\csname #1Change\endcsname{\CT@change[#2]}%
    \expandafter\def\csname #1Comment\endcsname{\CT@comment[#2]}%
    \expandafter\def\csname #1CommentTo\endcsname{\CT@commentto[#2]}%
    \expandafter\def\csname #1InlineComment\endcsname{\CT@inlinecomment[#2]}%
    \expandafter\def\csname #1InlineCommentTo\endcsname{\CT@inlinecommentto[#2]}%
  }%
}%

\let\newCTcontributor\CT@contributor

\newcommandx{\TODOthiswillnotshow}[2][1=]{\todo[disable,#1]{#2}}%


% Arguments:
% 1. (Optional) Contributor
% 2. Add text
\newcommand{\CT@add}[2][]{%
  \chgtrk@ifstrequal{#1}{}{%
    % Do nothing
  }{%
    \todo[linecolor=orange,backgroundcolor=orange!25,bordercolor=orange]{\textbf{#1} added}%{}
  }%
  \texthl{#2}%
}%

\newcommand{\CTadd}[1]{%
  \CT@add{#1}%
}%

% Arguments:
% 1. (Optional) Contributor
% 2. Add text
\newcommand{\CT@delete}[2][]{%
  \chgtrk@ifstrequal{#1}{}{%
    % Do nothing
  }{%
    \todo[linecolor=red,backgroundcolor=red!25,bordercolor=red]{\textbf{#1} delete}%{}
  }%
  \setstcolor{red}%
  \st{#2}%
}%

\newcommand{\CTdelete}[1]{%
  \CT@delete{#1}%
}%

\newcommand{\CT@change}[3][]{%
  \setstcolor{red}%
  \st{#2}%
  \chgtrk@ifstrequal{#1}{}{%
    % Do nothing
  }{%
    \todo[linecolor=blue,backgroundcolor=blue!25,bordercolor=blue]{\textbf{#1} changed}%{}
  }%
  \texthl{#3}%
  %\todo[linecolor=blue,backgroundcolor=blue!25,bordercolor=blue]{\textbf{Before}:
  %}%
}%

\newcommand{\CTchange}[2]{%
  \CT@change{#1}{#2}%
}%

\newcommand{\CT@comment}[2][]{%
  \chgtrk@ifstrequal{#1}{}{%
    \todo[linecolor=green,backgroundcolor=green!25,bordercolor=green]{#2}%{}
  }{%
    \todo[linecolor=green,backgroundcolor=green!25,bordercolor=green]{\textbf{#1}: #2}%{}
  }%
}%

\newcommand{\CT@commentto}[3][]{%
  \chgtrk@ifstrequal{#1}{}{%
    \todo[linecolor=green,backgroundcolor=green!25,bordercolor=green]{$\rightarrow$\textbf{#2}
      #3}%
  }{%
    \todo[linecolor=green,backgroundcolor=green!25,bordercolor=green]{\textbf{#1}:
      $\rightarrow$\textbf{#2} #3}%
  }%
}%

\newcommand{\CTcomment}[1]{%
  \CT@comment{#1}%
}%

\newcommand{\CTcommentto}[2]{%
  \CT@commentto{#1}{#2}%
}%

\newcommand{\CT@inlinecomment}[2][]{%
  \chgtrk@ifstrequal{#1}{}{%
    \todo[inline, linecolor=green,backgroundcolor=green!25,bordercolor=green]{#2}%{}
  }{%
    \todo[inline, linecolor=green,backgroundcolor=green!25,bordercolor=green]{\textbf{#1}: #2}%{}
  }%
}%

\newcommand{\CT@inlinecommentto}[3][]{%
  \chgtrk@ifstrequal{#1}{}{%
    \todo[inline,
    linecolor=green,backgroundcolor=green!25,bordercolor=green]{$\rightarrow$\textbf{#2}
      #3}%
  }{%
    \todo[inline,
    linecolor=green,backgroundcolor=green!25,bordercolor=green]{\textbf{#1}:
      $\rightarrow$\textbf{#2} #3}%
  }%
}%

\newcommand{\CTinlinecomment}[1]{%
  \CT@inlinecomment{#1}%
}%

\newcommand{\CTinlinecommentto}[2]{%
  \CT@inlinecommentto{#1}{#2}%
}%

%%%%% BEGIN Declaration of options %%%%%
% ========================
\DeclareOption{disabled}{%
  \renewcommand{\CT@add}[2][]{%
    #2% Display only the added text
  }%
  \renewcommand{\CT@delete}[2][]{}% Do nothing
  \renewcommand{\CT@change}[3][]{%
    #3% Display only the changed text
  }%
  \renewcommand{\CT@comment}[2][]{}% Do nothing
  \renewcommand{\CT@inlinecomment}[2][]{}% Do nothing
}%

%%%%% END Declaration of options %%%%%
% ========================

%%%%% BEGIN Execution of options %%%%%
% ========================

\ProcessOptions
%%%%% END Execution of options %%%%%

%    \end{macrocode}
% \end{macro} ^^A \newabbrev
%
%
% \Finale
\endinput